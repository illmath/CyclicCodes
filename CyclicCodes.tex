\documentclass[letterpaper, 12pt]{article}

\usepackage[utf8]{inputenc}
\usepackage[english]{babel}

\usepackage{geometry}
 \geometry{
 a4paper,
 total={170mm,257mm},
 left=1in,
 top=1in,
 }

 \usepackage{amssymb}
 \usepackage{amsmath}
 \usepackage{amsthm}

\newtheorem{theorem}{Theorem}[section]
\newtheorem{corollary}[theorem]{Corollary}
\newtheorem{lemma}[theorem]{Lemma}
\newtheorem{defn}[theorem]{Definition}
\title{Cyclic Codes}
\author{Kyle Cook \& Jacob Hauck}
\date{November 7, 2019}

\begin{document}

\maketitle

\section{Review of Ideals}
\defn Let $R$ be a ring with operations $+$ and $\cdot$. An ideal $I$ of $R$ is a subset of $R$ satisfying the following properties:
  \begin{enumerate}
    \item $I$ is a subgroup of $R$ under $+$
    \item for any $r\in R$ and any $i\in I$, $ri\in I$
  \end{enumerate}
\defn Let $R$ be a ring and $I$ a two sided ideal of $R$.  We can define an equivalence relation $\sim$ on $R$ as follows:
\[a\sim b \iff a-b \in I\]
The equivalence class of the element $a$ in $R$ is given by
\[\left[a\right] = a + I := \left\{a + r | r \in I\right\}\]
The set of all equivalence classes is denoted $R/I$; it becomes a ring, the factor ring, or quotient ring of $R$ modulo $I$, if one defines
\[(a+I) + (b+I) = (a+b) + I\]
\[(a+I)(b+I) = (ab) + I\]
In practice one must check these definitions are well defined.

\defn Let $a\in R$.  The set $\left<a\right> = \left\{ra | r \in R\right\}$ is an ideal of $R$ generated by $a$.  Ideals with such a generator element are called Principal Ideals.

\defn An integral domain is a nonzero commutative ring in which the product of any two nonzero elements is nonzero.

\theorem In an integral domain, every nonzero element $a$ has the cancellation property, that is, if $a\neq0$, then $ab=ac \implies b=c$

\defn A principal ideal domain is an integral domain in which every ideal is a principal ideal.

\defn $I$ is a maximal ideal of a ring $R$ if there are no other ideals contained between $I$ and $R$.

\theorem Given a ring $R$ and a proper ideal $I$ of $R$, that is $I\neq R$, $I$ is a maximal ideal of $R$ if any of the following equivalent conditions hold:
\begin{enumerate}
  \item There exists no other proper ideal $J$ or $R$ so that $I \subset J$.
  \item For any ideal $J$ with $I\subseteq J$, either $J=I$ or $J=R$.
  \item The quotient ring $R/I$ has no nontrivial ideals.
\end{enumerate}

\defn Given a field $\mathbb{F}$ we define the ring of polynomials in $x$ over $\mathbb{F}$, $\mathbb{F}[x]$, as the set of all polynomials $p = p_0 + p_1x + p_2x^2 + \cdots p_kx^k$ where $p_i$ are coefficients in $\mathbb{F}$


\end{document}
